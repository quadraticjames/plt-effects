% Template for a Computer Science Tripos Part II project dissertation
\documentclass[12pt,a4paper,twoside,openright]{report}
\usepackage[pdfborder={0 0 0}]{hyperref}    % turns references into hyperlinks
\usepackage[margin=25mm]{geometry}  % adjusts page layout
\usepackage{graphicx}  % allows inclusion of PDF, PNG and JPG images
\usepackage{verbatim}

\raggedbottom                           % try to avoid widows and orphans
\sloppy
\clubpenalty1000%
\widowpenalty1000%

\renewcommand{\baselinestretch}{1.1}    % adjust line spacing to make
                                        % more readable

\begin{document}

\bibliographystyle{plain}


%%%%%%%%%%%%%%%%%%%%%%%%%%%%%%%%%%%%%%%%%%%%%%%%%%%%%%%%%%%%%%%%%%%%%%%%
% Title


\pagestyle{empty}

\rightline{\LARGE \textbf{James Wright}}

\vspace*{60mm}
\begin{center}
\Huge
\textbf{Formalising the semantics of algebraic effects in OCaml} \\[5mm]
Computer Science Tripos -- Part II \\[5mm]
Trinity Hall \\[5mm]
\today  % today's date
\end{center}

%%%%%%%%%%%%%%%%%%%%%%%%%%%%%%%%%%%%%%%%%%%%%%%%%%%%%%%%%%%%%%%%%%%%%%%%%%%%%%
% Proforma, table of contents and list of figures

\pagestyle{plain}

\chapter*{Proforma}

{\large
\begin{tabular}{p{4.4cm}p{11cm}}
Name:               & \bf James Wright                       \\
College:            & \bf Trinity Hall                     \\
Project Title:      & \bf Formalising the semantics of algebraic effects in OCaml \\
Examination:        & \bf Computer Science Tripos -- Part II, 2015--16  \\
Word Count:         & \bf unknown  \\
Project Originator: & KC Sivaramakrishnan                    \\
Supervisor:         & KC Sivaramakrishnan                    \\ 
\end{tabular}
}


\section*{Original Aims of the Project}

To formalise the semantics of algebraic effects, a proposed addition to OCaml, as a CEK machine, in order to perform subsequent random and exhaustive testing of concurrent schedules in the program. The testing should allow us to show desirable properties such as determinism and absence of deadlock.

\section*{Work Completed}

\section*{Special Difficulties}

None.
 
\newpage
\section*{Declaration}

I, James Wright of Trinity Hall, being a candidate for Part II of the Computer
Science Tripos, hereby declare
that this dissertation and the work described in it are my own work,
unaided except as may be specified below, and that the dissertation
does not contain material that has already been used to any substantial
extent for a comparable purpose.

\bigskip
\leftline{Signed}

\medskip
\leftline{Date}

\tableofcontents

\listoffigures

%%%%%%%%%%%%%%%%%%%%%%%%%%%%%%%%%%%%%%%%%%%%%%%%%%%%%%%%%%%%%%%%%%%%%%%
% now for the chapters

\pagestyle{headings}

\chapter{Introduction}

\section{Algebraic effects}

There exist multicore-capable functional programming languages, such as Haskell, Manticore, F\#, and MultiMLton, whose runtimes provide support for concurrency---the concurrency primitives are baked in to the runtime system. This has the advantage of natively supporting multicore capabilities, but it also means that the runtimes can become burdensomely complicated. OCaml, on the other hand, is not yet multicore-capable and does not have this concurrency support.
Algebraic effects is a proposed addition to OCaml that would provide this. In contrast to the implementations in the other aforementioned functional programming languages, algebraic effects provide a modular abstraction for expressing effectful computation, which allows programmers to implement independent schedulers and new concurrency primitives as OCaml libraries.

\section{Formalisation}

The syntax and semantics for algebraic effects have been designed and implemented by OCaml Labs, but the new semantics has not yet been formalised. Formalisation allows for random and exhaustive testing of concurrent schedules, which would allow us to show properties such as determinism or absence of deadlock.

\chapter{Preparation}



\chapter{Implementation}

\chapter{Evaluation}

\chapter{Conclusion}


%%%%%%%%%%%%%%%%%%%%%%%%%%%%%%%%%%%%%%%%%%%%%%%%%%%%%%%%%%%%%%%%%%%%%
% the appendices
\appendix

\chapter{Project Proposal}

\end{document}
